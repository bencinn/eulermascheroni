% In this file you should put the actual content of the blueprint.
% It will be used both by the web and the print version.
% It should *not* include the \begin{document}
%
% If you want to split the blueprint content into several files then
% the current file can be a simple sequence of \input. Otherwise It
% can start with a \section or \chapter for instance.

\begin{lemma}[Nonnegativity of the inner sum]
  \label{lem:eulerMascheroni_inner_nonneg}
  \lean{EulerMascheroniInfiniteSum.eulerMascheroni_sum_inner_nonneg}
  \leanok
  For every integer \(n \ge 1\),
  \[
    \frac{1}{n} - \ln\!\left(\frac{n+1}{n}\right) \ge 0.
  \]
\end{lemma}

\begin{proof}
  The inequality \(\ln x \le x-1\) holds for all \(x>0\). Taking
  \(x=\dfrac{n+1}{n}>0\) gives
  \[
    \ln\!\left(\frac{n+1}{n}\right) \le \frac{n+1}{n}-1 = \frac{1}{n}.
  \]
  Rearranging yields the stated nonnegativity.
\end{proof}

\begin{lemma}[Partial sum of the infinite sum]
  \label{lem:eulerMascheroni_tsum_partial}
  \lean{EulerMascheroniInfiniteSum.eulerMascheroni_sum_partial}
  \leanok
  For every integer \(N \ge 1\),
  \[
    \sum_{n=1}^{N}\left(\frac{1}{n} - \ln\!\left(\frac{n+1}{n}\right)\right)
    = H_{N} - \ln\!(N+1),
  \]
  where \(H_{N}=\sum_{n=1}^N \frac{1}{n}\) is the \(N\)-th harmonic number.
\end{lemma}

\begin{proof}
  Split the sum:
  \[
    \sum_{n=1}^{N}\frac{1}{n} - \sum_{n=1}^{N}\ln\!\left(\frac{n+1}{n}\right).
  \]
  The first term is \(H_{N}\). The second term telescopes:
  \[
    \sum_{n=1}^{N}\left(\ln(n+1)-\ln n\right)=\ln(N+1)-\ln 1=\ln(N+1).
  \]
  Combining gives the claimed identity.
\end{proof}

\begin{theorem}[Sondow 2005 (2) (Infinite sum)]
  \label{thm:eulerMascheroni_tsum}
  \lean{EulerMascheroniInfiniteSum.eulerMascheroni_tsum}
  \uses{lem:eulerMascheroni_inner_nonneg, lem:eulerMascheroni_tsum_partial}
  \leanok
  \[
    \gamma = \lim_{n \to \infty}\!\left(H_{n} - \ln n\right)
      = \sum_{n=1}^{\infty}\left(\frac{1}{n} - \ln\!\left(\frac{n+1}{n}\right)\right).
  \]
  (Note: in the Lean definition the index of the inner summand is shifted
  by one so the series is expressed as a sum from \(n=0\) to \(\infty\).)
\end{theorem}
